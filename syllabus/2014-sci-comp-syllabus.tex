
\documentclass[11pt,letterpaper]{article}

\usepackage[T1]{fontenc}
\usepackage[sc]{mathpazo}

\usepackage{indentfirst}   %% if you want to have the first paragraph of each section indented %%
\usepackage[top=0.75in, bottom=0.75in, left=1in, right=1in]{geometry} %% for setting page margins %%
\usepackage{hyperref}
\hypersetup{
colorlinks=true
}
\usepackage{titlesec} % for more compact section spacing
\titlespacing{\section}{0pt}{*2}{*1}

\begin{document}


\section*{\centering Biology 723: Scientific Computing for Biologists, Fall 2014}
\begin{center}
\begin{tabular}{ll}
\textbf{Instructor}: Paul M. Magwene & \textbf{Office}: FFSC 4103\\
\textbf{Phone}: 613-8159 & \textbf{Email}: paul.magwene@duke.edu\\
&\\
\textbf{TA}: Yuantong Ding & \textbf{Office}: TBA\\
\textbf{Phone}: 919-457-7658 & \textbf{Email}: yuantong.ding@duke.edu\\
\end{tabular}
\end{center}

\section*{Description}

The focus of this course is statistical computing for the biological sciences with an emphasis on common multivariate statistical methods and techniques for exploratory data analysis. A major goal of the course is to help graduate students in the biological sciences develop practical insights into methods that they are likely to encounter in their own research, and the potential advantages and pitfalls that come with their use.

\section*{Prerequisites}

Enrollment is limited to graduate students or undergraduates with permission of instructor. No previous programming experience is required, but familiarity with basic statistical concepts (equivalent of STA 213) is assumed.

\section*{Grading}
Grading is based on weekly homework assignments. These homework assignments will typically consist of statistical problem solving exercises and/or programming tasks.

\section*{Office Hours}
Yuantong will hold office hours on Wednesdays from 2-4pm (location TBA).

\section*{Course Website}

\href{https://github.com/pmagwene/Bio723}{https://github.com/pmagwene/Bio723/wiki}

\renewcommand{\refname}{Texts}
\begin{thebibliography}{99}

\bibitem{Wickens} Wickens, T.\ D. 1995. The geometry of multivariate statistics. Lawrence Earlbaum Associates, New Jersey.

\bibitem{CSpy}A. B. Downey. 2014. Think Python: How to think like a computer scientist. Available from \href{http://www.greenteapress.com/thinkpython/thinkpython.html}{Greentea Press} under an open source license.

\bibitem{Matloff} Matloff, N. 2011. The Art of R Programming. No Starch Press, San Francisco.



\end{thebibliography}

\renewcommand{\refname}{Other Recommended Texts}

\begin{thebibliography}{99}
\setcounter{enumiv}{5}
\bibitem{Hamilton} Hamilton, A.\ G. 1989. Linear algebra: An introduction with concurrent examples. Cambridge University Press, Cambridge.

A well organized and readable introduction to linear algebra. This assumes no previous familiarity with lineage algebra.  You'll get maximum benefit from this text if you work through the short exercises that accompany each chapter.

\bibitem{Krzanowski} Krzanowski, W.\ J. 2000. Principles of multivariate analysis. Oxford Univ. Press, New York.

I would have made this a required text but it's become unreasonably expensive, even for used copies. Nonetheless, if you plan on having just on book on multivariate statistics on your bookshelf this is the one I'd recommend.

\bibitem{Sokal} Sokal, R.\ R. and F.\ J. Rohlf. 1995. Biometry. W.\ H. Freeman, New York.

Another good text to have on your bookshelf. A readable and well organized basic statistics book with examples drawn from the biological literature.

\end{thebibliography}

\section*{Syllabus}

\renewcommand{\arraystretch}{1.4}
\begin{center}
\begin{tabular}{rp{5.5in}}
\multicolumn{1}{c}{{\sl Date}} & \multicolumn{1}{c}{{\sl Topic}} \\

August 26 & Introduction; Getting Acquainted with Python and Unix \\
September 2 & Data as Vectors; Uni- and bivariate visualizations\\
September 9 & Descriptive statistics as matrix operations; Visualizing multivariate data\\
September 16 & Multiple regression and introduction to biplots; Regression\\
September 23 & Non-linear regression models\\
September 30 & Eigenvectors and Eigenvalues; Principal Components Analysis \\
October 7 & Singular Value Decompisition and Biplots [RESCHEDULE MAKEUP CLASS]\\
October 14 & \multicolumn{1}{c}{{\sc Fall Break}} \\
October 21 & Discriminant analysis and Canonical Variate Analysis\\
October 28 & Analyses based on Similarity/Distance I; Hierarchical and K-means clustering\\
November 4 & Analyses based on Similarity/Distance II; Multidimensional scaling\\
November 11 & Randomization and Monte Carlo Methods; Jackknife, Bootstrap, Permutation\\
November 18 & Building Bioinformatics Pipelines I; Pipes, redirection, subprocesses \\
November 25 & Building Bioinformatics Pipelines II; Putting the concepts to work \\

& \multicolumn{1}{c}{{\sc Graduate Classes End November 25}} \\
\end{tabular}
\end{center}



\end{document}
