

\section{Elementary data structures in Python}

\subsection{Python Lists}

Lists are the simplest `built-in' data structure in Python. List
represent ordered collections of arbitrary objects.
%
\begin{python}
>>> l = [2, 4, 6, 8, 'fred']
>>> l
[2, 4, 6, 8, 'fred']
>>> len(l)
5
\end{python}

Python lists are zero-indexed. This means you can access lists elements
|0| to |len(x)-1|.
%
\begin{python}
>>> l[0]
2
>>> l[3]
8
>>> l[5]
Traceback (most recent call last):
  File "<stdin>", line 1, in <module>
IndexError: list index out of range
\end{python}
%
You can use negative indexing to get elements from the end of the list:
\begin{python}
>>> l[-1] # the last element
'fred'
>>> l[-2] # the 2nd to last element
8
>>> l[-3] # ... etc ...
6
\end{python}

Indexing can be used to both get and set items in a list.
\begin{python}
>>> l = [2, 4, 6, 8, 'hike!']
>>> l[-1]
'hike!'
>>> l[-1] = "learning python is so great!"
>>> l
[2, 4, 6, 8, 'learning python is so great!']
\end{python}

Python lists support the notion of `slices' - a continuous sublist of a
larger list. The following code illustrates this concept:
%
\begin{python}
>>> y = range(10)  # our first use of a function!
>>> y
[0, 1, 2, 3, 4, 5, 6, 7, 8, 9]
>>> y[2:8]
[2, 3, 4, 5, 6, 7]
>>> y[2:-1] # the slice
[2, 3, 4, 5, 6, 7, 8]
>>> y[-1:0] # how come this didn't work? 
[]
# slice from last to first, stepping backwards by 2
>>> y[-1:0:-2]  
[9, 7, 5, 3, 1]
\end{python}
%
As with single indexing, the slice notation can be used to set elements of a list as well:
\begin{python}
>>> s = ['a', 'b', 'c', 'd', 'e']
>>> s[2:4] = ['C','D']
>>> s
['a', 'b', 'C', 'D', 'e']
\end{python}

You can also append and delete list elements as well as concatenate two lists:
\begin{python}
>>> l1 = [1,2,3]
>>> l2 = ['a','b','c','d']
>>> l1.append(4)
>>> l1
[1, 2, 3, 4]
>>> del l2[2]
>>> l2
['a', 'b', 'd']
>>> l3 = l1 + l2
>>> l3
[1, 2, 3, 4, 'a', 'b', 'd']
\end{python}
%
Finally, there are a number of useful methods associated with list objects, such as |reverse()| and |sort()|:
\begin{python}
>>> l4 = [1, 5, 3, 4, 10, 11, 3]
>>> l4.sort() # sort in-place
>>> l4
[1, 3, 3, 4, 5, 10, 11]
>>> l4.reverse() # reverse in-place
>>> l4
[11, 10, 5, 4, 3, 3, 1]
\end{python}

Read, `Think Python' \href{http://www.greenteapress.com/thinkpython/html/thinkpython011.html}{Chapter 10} for more about lists.  See the \href{http://docs.python.org/2.7/library/stdtypes.html#sequence-types-str-unicode-list-tuple-bytearray-buffer-xrange}{Python library documentation} for a summary of methods associated with lists.


\subsection{Python Tuples}

As shown above, the elements of a Python list are mutable and the length of lists can be changed.  Sometimes it's useful for reasons of both programming logic and computational efficiency to have immutable collections of items.  The data structure that Python uses to represent such immutable collection is called a `Tuple'.

Parentheses rather than square brackets are used to create tuples.  A common use of tuples is to represent a pair, i.e. a tuple of length two.  For example, you might use a pair to represent a point in the Cartesian plane:
%
\begin{python}
>>> pt = (1, 5)
>>> pt
(1, 5)
>>> pt[0]
1
>>> pt[1]
5
>>> pt[0] = 2
Traceback (most recent call last):
  File "<ipython-input-11-648d2cfb62f4>", line 1, in <module>
    pt[0] = 2
TypeError: 'tuple' object does not support item assignment
\end{python}
%
The above illustrates that tuples can be indexed like lists, however you can't assign to or extend a tuple after it's created. As we will illustrate later, tuples are often used to return multiple objects from functions.

Read, `Think Python' \href{http://www.greenteapress.com/thinkpython/html/thinkpython013.html}{Chapter 12} for more about tuples.



\subsection{Python Dictionaries}

Python dictionaries (sometimes called `hash maps' in other programming languages) map a set of keys (labels) to values.  Dictionaries provide efficient access to non-ordered data.
%
\begin{python}
## create a dictionary
>>> d = {'fred':26, 'jim':22, 'mary':34, 'jill': 12}
>>> d
{'jim': 22, 'mary': 34, 'jill': 12, 'fred': 26}
>>> d['fred']  ## get value associated with the key 'fred'
26
>>> d.keys()  # return list of keys in dict
['jim', 'mary', 'jill', 'fred']
>>> d.values()  # return list of values in dict
[22, 34, 12, 26]
>>> for key in d.keys():    # iterate over the keys in the dict
...     print key, "is", d[key], "years old."
...     
jim is 22 years old.
mary is 34 years old.
jill is 12 years old.
fred is 26 years old.
>>> d['joe'] = 99  # add a new key,value pair to dict
>>> d
{'joe': 99, 'jim': 22, 'mary': 34, 'jill': 12, 'fred': 26}
>>> del d['mary']  # remove a key,value pair from dict
>>> d
{'joe': 99, 'jim': 22, 'jill': 12, 'fred': 26}
\end{python}

Dictionary keys have to be immutable objects, but associated values can be arbitrary python objects, even other dictionaries.
%
\begin{python}
>>> d2 = {}   # create an empty dict
>>> d2[[1,2,3]] = 'b'   # a list can't be a key
Traceback (most recent call last):
  File "<ipython-input-52-497f3cfcc4ef>", line 1, in <module>
    d2[[1,2,3]] = 'b'
TypeError: unhashable type: 'list'

>>> d2[(1,2,3)] = 'b'  # but a tuple works
>>> d2
{(1, 2, 3): 'b'}
>>> d3 = dict()  # alternate way to create empty dict
# create a dictionary of dictionaries
>>> d3['fred'] = {'age': 26, 'sex': 'male', 'occupation':'nurse'}
>>> d3['mary'] = {'age': 34, 'sex': 'female', 'occupation':'engineer'}
>>> d3['fred']['age']
26
>>> d3['mary']['occupation']
'engineer'
\end{python}

Read, `Think Python' \href{http://www.greenteapress.com/thinkpython/html/thinkpython012.html}{Chapter 11} for more about dictionaries.
