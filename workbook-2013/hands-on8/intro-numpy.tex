
\section{Introduction to \numpy}

As mentioned during lecture, Python does not have a built-in data
structure that behaves in quite the same way as do vectors in R.
However, we can get very similar behavior using a library called \numpy.

\numpy does not come with the standard Python distribution, but it does come
as an included package if you use the Enthought Python distribution.
Alternately you can download \numpy from the SciPy project page at:
\url{http://numpy.scipy.org}. The \numpy package comes with documentation and
a tutorial. You can access the documentation at
\url{http://docs.scipy.org/doc/}.

Here's some examples illustrating using of \numpy:
\begin{python}
>>> from numpy import array # a third form of import 
>>> x = array([2,4,6,8,10])
>>> -x
array([ -2,  -4,  -6,  -8, -10])
>>> x ** 2
array([  4,  16,  36,  64, 100])
>>> pi * x # assumes pi is in the current namespace
array([  6.28318531,  12.56637061,  18.84955592,  25.13274123,  31.41592654])
>>> y = array([0, 1, 3, 5, 9])
>>> x + y
array([ 2,  5,  9, 13, 19])
>>> x * y
array([ 0,  4, 18, 40, 90])
>>> z = array([1, 4, 7, 11])
>>> x+z
Traceback (most recent call last):
  File "<stdin>", line 1, in <module>
ValueError: shape mismatch: objects cannot be broadcast to a single shape
\end{python}
%
The last example above shows that, unlike R, \numpy arrays in Python are
not `recycled' if lengths do not match.

Remember that lists and arrays in Python are zero-indexed rather than
one-indexed.
%
\begin{python}
>>> x
array([ 2,  4,  6,  8, 10])
>>> len(x)
5
>>> x[0]
2
>>> x[1]
4
>>> x[4]
10
>>> x[5]

Traceback (most recent call last):
  File "<pyshell#52>", line 1, in -toplevel-
    x[5]
IndexError: index out of bounds
\end{python}

\numpy arrays support the comparison operators and return arrays of
booleans.
\begin{python}
    >>> x < 5 
    array([ True, True, False, False, False], dtype=bool)
    >>> x >= 6 
    array([0, 0, 1, 1, 1])
\end{python}
%
\numpy also supports the combination of comparison and indexing that R vectors
can do. There are also a variety of more complicated indexing functions
available for NumPy; see the
\href{http://docs.scipy.org/doc/numpy/reference/routines.indexing.html}{Indexing Routines} in the Numpy docs.
%
\begin{python}
>>> x[x < 5]
array([2, 4])
>>> x[x >= 6]
array([ 6,  8, 10])
>>> x[(x<4)+(x>6)]  # 'or'
array([ 2,  8, 10])
\end{python}
%
Note that Boolean addition is equivalent to `or' and Boolean
multiplication is equivalent to `and'.

Most of the standard mathematical functions can be applied to NumPy
arrays however you must use the functions defined in the
\numpy module.
%
\begin{python}
>>> x
array([ 2,  4,  6,  8, 10])
>>> import math
>>> math.cos(x)

Traceback (most recent call last):
  File "<pyshell#67>", line 1, in -toplevel-
    math.cos(x)
TypeError: only length-1 arrays can be converted to Python scalars.
>>> import numpy
>>> numpy.cos(x)
array([-0.41614684, -0.65364362,  0.96017029, -0.14550003, -0.83907153])
\end{python}





\subsection{Vector Operations in \numpy}

The Python equivalent of the R code above is:
%
\begin{python}
>>> import numpy
>>> x = numpy.arange(start=1, stop=16, step=1)
>>> y = numpy.arange(10,25) # default step = 1
>>> x
array([ 1,  2,  3,  4,  5,  6,  7,  8,  9, 10, 11, 12, 13, 14, 15])
>>> y
array([10, 11, 12, 13, 14, 15, 16, 17, 18, 19, 20, 21, 22, 23, 24])
>>> x+y
array([11, 13, 15, 17, 19, 21, 23, 25, 27, 29, 31, 33, 35, 37, 39])
>>> x-y
array([-9, -9, -9, -9, -9, -9, -9, -9, -9, -9, -9, -9, -9, -9, -9])
>>> 3*x
array([ 3,  6,  9, 12, 15, 18, 21, 24, 27, 30, 33, 36, 39, 42, 45])
>>> z = numpy.dot(x,x) # no built-in dot operator, but a dot fxn in numpy
>>> z
1240
\end{python}
%
Note the use of the \lstinline!numpy.arange()! function.
\lstinline!numpy.arange()! works like R's \lstinline!sequence()!
function and it returns a Numpy array. However, notice that the values
go up to but don't include the specified \lstinline!stop! value. Use
\lstinline!help()! to lookup the documentation for
\lstinline!numpy.arange()!. Python also includes a \lstinline!range()!
function that generates a regular sequence as a Python list object. The
\lstinline!range()! function has \lstinline!start!, \lstinline!stop!,
and \lstinline!step! arguments but these can only be integers. Here are
some additional examples of the use of \lstinline!arange()! and
\lstinline!range()!:

\begin{python}
>>> z = numpy.arange(1,5,0.5) 
>>> z
array([ 1. ,  1.5,  2. ,  2.5,  3. ,  3.5,  4. ,  4.5])
>>> range(1,20,2)
[1, 3, 5, 7, 9, 11, 13, 15, 17, 19]
>>> range(1,5,0.5)
Traceback (most recent call last):
  File "<stdin>", line 1, in <module>
TypeError: range() integer step argument expected, got float.
\end{python}



\subsection{Matrices in \numpy}

Matrices in Python are created are created using the
\lstinline!Numeric.array()! function. In Python you need to be a little
more aware of the type of the arrays that you create. If the argument
you pass to the \lstinline!array()! function is composed only of
integers than Numeric will assume you want an integer matrix which has
consequences in terms of operations like those illustrated below. To
make sure you're matrix has floating type values you can use the
argument \lstinline!typecode=Numeric.Float!.

\begin{python}
>>> import numpy as np # I'm 'aliasing' the name so I can type 'np' instead of 'numpy'
>>> array = np.array # setup another alias
>>> X = array(range(1,13))
>>> X
array([ 1,  2,  3,  4,  5,  6,  7,  8,  9, 10, 11, 12])
>>> X.shape = (4,3) # rows, columns
>>> X
array([[ 1,  2,  3],
       [ 4,  5,  6],
       [ 7,  8,  9],
       [10, 11, 12]])
>>> 1/X # probably not what you expected
array([[1, 0, 0],
       [0, 0, 0],
       [0, 0, 0],
       [0, 0, 0]])
>>> X = array(range(1,13), dtype=np.float)
>>> X.shape = 4,3
>>> X
array([[  1.,   2.,   3.],
       [  4.,   5.,   6.],
       [  7.,   8.,   9.],
       [ 10.,  11.,  12.]])
>>> 1/X # that's more like it
array([[ 1.        ,  0.5       ,  0.33333333],
       [ 0.25      ,  0.2       ,  0.16666667],
       [ 0.14285714,  0.125     ,  0.11111111],
       [ 0.1       ,  0.09090909,  0.08333333]])
>>> X
array([[  1.,   2.,   3.],
       [  4.,   5.,   6.],
       [  7.,   8.,   9.],
       [ 10.,  11.,  12.]])
>>> X + X
array([[  2.,   4.,   6.],
       [  8.,  10.,  12.],
       [ 14.,  16.,  18.],
       [ 20.,  22.,  24.]])
>>> X - X
array([[ 0.,  0.,  0.],
       [ 0.,  0.,  0.],
       [ 0.,  0.,  0.],
       [ 0.,  0.,  0.]])
>>> np.dot(X,np.transpose(X)) # dot fxn in numpy gives matrix multiplication for arrays
array([[  14.,   32.,   50.,   68.],
       [  32.,   77.,  122.,  167.],
       [  50.,  122.,  194.,  266.],
       [  68.,  167.,  266.,  365.]])
>>> np.identity(4)
array([[1, 0, 0, 0],
       [0, 1, 0, 0],
       [0, 0, 1, 0],
       [0, 0, 0, 1]])
>>> np.sqrt(X)
array([[ 1.        ,  1.41421356,  1.73205081],
       [ 2.        ,  2.23606798,  2.44948974],
       [ 2.64575131,  2.82842712,  3.        ],
       [ 3.16227766,  3.31662479,  3.46410162]])
>>> np.cos(X)
array([[ 0.54030231, -0.41614684, -0.9899925 ],
       [-0.65364362,  0.28366219,  0.96017029],
       [ 0.75390225, -0.14550003, -0.91113026],
       [-0.83907153,  0.0044257 ,  0.84385396]])
\end{python}

The code above also demonstrated the Numpy functions \lstinline!dot()!,
\lstinline!transpose()! and \lstinline!identity()!. Note too that Numpy
has a variety of functions such as \lstinline!sqrt()!and
\lstinline!cos()! that work on an element-wise basis.

Indexing of arrays in Numpy is demonstrated below. You'll see that
Python arrays support `slicing' operations. For more on slicing and
other array basics see the Numpy documentation at
\href{http://docs.scipy.org/doc/}{http://docs.scipy.org/doc/}.

\begin{python}
>>> X
array([[  1.,   2.,   3.],
       [  4.,   5.,   6.],
       [  7.,   8.,   9.],
       [ 10.,  11.,  12.]])
>>> X[0,0] # get the 0th row, 0th column (remember that Python sequences are zero-indexed!)
1.0
>>> X[3,0] # get the fourth row, 1st column
10.0
>>> X[:2,:2]  # an example of slicing, get the first two columns and rows (i.e. indices 0 and 1)
array([[ 1.,  2.],
       [ 4.,  5.]])
>>> X[1:,:2] # get everything after the 0th row and  the first two columns
array([[  4.,   5.],
       [  7.,   8.],
       [ 10.,  11.]])
\end{python}
To calculate matrix inverses in Python you need to import the
\lstinline!numpy.linalg! package.

\begin{python}
>>> import numpy.linalg as la
>>> import numpy.random as ra  # for matrices with elements from random distributions
>>> C = ra.normal(loc=0,scale=1,size=(4,4)) # do help(ra.normal) for explanation of argumnets
>>> C
array([[ 0.79525679,  1.11730719, -2.19257712, -0.06289276],
       [ 0.7087366 ,  0.70574975, -1.51599336, -0.90360945],
       [-0.33845153, -0.20109722, -0.75245988, -0.56027025],
       [-0.51692665,  0.59972543,  1.55562234,  1.88639367]])
>>> Cinv = la.inv(C)
>>> np.dot(C, Cinv) # again result is approx the identity matrix due to floating point precision
array([[ 1.00000000e+000, -5.55111512e-017, -6.93889390e-017,  2.94902991e-017],
       [ 1.11022302e-016,  1.00000000e+000, -1.11022302e-016, -5.55111512e-017],
       [ 1.11022302e-016, -2.22044605e-016,  1.00000000e+000,  2.77555756e-017],
       [ 0.00000000e+000, -4.44089210e-016,  0.00000000e+000,  1.00000000e+000]])
>>> print np.array2string(np.dot(C,Cinv),precision=2, suppress_small=True)
[[ 1. -0.  0.  0.]
 [-0.  1.  0.  0.]
 [ 0.  0.  1.  0.]
 [-0. -0. -0.  1.]]
\end{python}




