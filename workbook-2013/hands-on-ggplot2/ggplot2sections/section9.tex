% Created 2013-08-21 Wed 17:11
\documentclass[11pt]{article}
\usepackage[utf8]{inputenc}
\usepackage[T1]{fontenc}
\usepackage{fixltx2e}
\usepackage{graphicx}
\usepackage{longtable}
\usepackage{float}
\usepackage{wrapfig}
\usepackage{soul}
\usepackage{textcomp}
\usepackage{marvosym}
\usepackage{wasysym}
\usepackage{latexsym}
\usepackage{amssymb}
\usepackage{hyperref}
\tolerance=1000
\providecommand{\alert}[1]{\textbf{#1}}

\title{themes}
\author{Colin Maxwell}
\date{21 August 2013}

\begin{document}

\maketitle

\setcounter{tocdepth}{3}
\tableofcontents
\vspace*{1cm}

ggplot gives you almost unlimited control over the nitty gritty
details of the non-data elements of you plot by changing options in the
`theme' function. The full list of options you can specify is found here:

\href{http://docs.ggplot2.org/current/theme.html}{http://docs.ggplot2.org/current/theme.html}

Each option controls the appearance of an aspect of the plot. As you see in the documentation, each argument takes a different
class. These range from strings to functions. For example, if the part
of the plot that you want to control is text (an axis label or a
title, for example), the type of argument that you need to call is the
function `element$_{\mathrm{text}}$'. This function in turn has arguments for all
parts of the appearance of the text (the size, orientation, etc.)
There's way more here than I want to go into so I'll just go over a couple of my favorites.

\section{Default themes}
\label{sec-1}


There's a philosophy behind the default ggplot theme (the grey
background is to not make too much whitespace which draws your eyes,
and the guidelines help you determine exactly where individual
datapoints are). 

\begin{verbatim}
ggplot( growth.270, aes(x=time,
                        y=OD,
                        col=strain,
                        group=plate_pos))+
    geom_line()+
    theme_grey()
\end{verbatim}


However, I feel like it's rather ugly. For single
facets, I like `theme$_{\mathrm{classic}}$,' which makes the plot look like more
traditional. It's also more compatible with greyscale color mappings:

\begin{verbatim}
ggplot( growth.270, aes(x=time,
                        y=OD,
                        col=strain,
                        group=plate_pos))+
    geom_line()+
    theme_classic()+
    scale_color_grey()
\end{verbatim}


However, it doesn't look as nice with facets, since it doesn't give
you the bounding box for the plots:

\begin{verbatim}
ggplot( growth.270, aes(x=time,
                        y=OD,
                        col=strain,
                        group=plate_pos))+
    geom_line()+
    facet_grid(dextrose~ammonium, margins = TRUE)+
    theme_classic()
\end{verbatim}


In this case, I prefer `theme$_{\mathrm{bw}}$'

\begin{verbatim}
ggplot( growth.270, aes(x=time,
                        y=OD,
                        col=strain,
                        group=plate_pos))+
    geom_line()+
    facet_grid(dextrose~ammonium, margins = TRUE)+
    theme_bw()
\end{verbatim}


These theme functions are a good place to understand how themes work
in ggplot. If you just type in the function name with no parentheses
you can see the code that defines them:

\begin{verbatim}
theme_grey
\end{verbatim}


The function wraps a call to the function `theme' which has a large
number of arguments that specifies how the plot looks. You can
subsequently call the same function `theme' to change of those
arguments. So, if we wanted the plot background to be a different
color, we could adjust it:

\begin{verbatim}
ggplot( growth.270, aes(x=time,
                        y=OD,
                        col=strain,
                        group=plate_pos))+
    geom_line()+
    facet_grid(dextrose~ammonium, margins = TRUE)+
    theme_classic()+
    theme(plot.background = element_rect(fill = "black"))
\end{verbatim}

Another nice thing is that you can easily define functions for later
use. We can get a clue how to do this by looking at the code for
theme$_{\mathrm{bw}}$:

\begin{verbatim}
theme_bw
\end{verbatim}


The function theme is invoked again, but this time with the funny
operator ``\%+replace+\%''. ``\%+replace+\%'' operates like a ``+'' sign or a
``-'' sign in order to perform an operation. In this case, what it does
is replace the aspects in the theme$_{\mathrm{grey}}$ that are specified in the
second call to theme. The advantage to doing this rather than just
calling theme directly is that it lets you keep control of the font
and the font size which are specified through the arguments
``base$_{\mathrm{size}}$'' and ``base$_{\mathrm{family}}$''

With this in mind, I made a function to generate plots that have that
dramatic black background that you see sometimes in presentations:

\begin{verbatim}
theme_dramatic_black <- function(base_size = 12, base_family = ""){
    theme_grey(base_size = base_size, base_family = base_family) %+replace%
        theme(
            panel.background = element_rect(fill = NA, color="white"),
            plot.background = element_rect(fill = "black"),
            axis.title.y=element_text(color="white"),
            axis.title.x=element_text(color="white"),
            axis.text.x=element_text(color="white"),
            axis.text.y=element_text(color="white"),
            legend.key=element_rect(color=NA, fill = NA),
            legend.text=element_text(color="white"),
            legend.background=element_rect(color="black", fill="black"),
            strip.background=element_rect(color="white", fill = NA),
            strip.text = element_text(color="white"),
            panel.grid.minor = element_line(color=NA),
            panel.grid.major = element_line(size = 0.1, color = "white")
            )
}

ggplot( growth.270, aes(x=time,
                          y=OD,
                          col=strain,
                          group=plate_pos))+
    geom_line()+
    facet_grid(dextrose~ammonium, margins = TRUE)+
    theme_dramatic_black()
\end{verbatim}


Since we wrapped the the function correctly, we can now easily adjust
the text size, and the font:

\begin{verbatim}
ggplot( growth.270, aes(x=time,
                          y=OD,
                          col=strain,
                          group=plate_pos))+
    geom_line()+
    facet_grid(dextrose~ammonium, margins = TRUE)+
    scale_x_continuous(label = comma)+
    theme_dramatic_black(base_size = 16, base_family = "Garamond")
\end{verbatim}


This function could then be easily called to apply the same settings
to every plot in our presentation.
\section{Axis labels and title}
\label{sec-2}


A very common task is to change the labels on the plot. You can either
do that using the `scale' functions as above, or with some dedicated
functions:

\begin{verbatim}
ggplot( growth.270, aes(x=time,
                        y=OD,
                        col=strain,
                        group=plate_pos))+
    geom_line()+
    facet_grid(dextrose~ammonium, margins = TRUE)+
    theme_bw()+
    ggtitle("My title")+
    ylab("Y axis title")+
    xlab("X axis title")
\end{verbatim}
\section{Text orientation}
\label{sec-3}




I often find that my x axis labels run into each other (especially
with facets). You can easily change this by setting the rotation and
the x axis justification:

\begin{verbatim}
ggplot( growth.270, aes(x=time,
                        y=OD,
                        col=strain,
                        group=plate_pos))+
    geom_line()+
    facet_grid(dextrose~ammonium, margins = TRUE)+
    theme_bw()+
    ggtitle("Gowth curves")+
    xlab("Time (seconds)")+
    theme(axis.text.x = element_text(angle=90, hjust = 1, vjust = 0.5))
\end{verbatim}


The options available for the axis text are options passed to the
function `element$_{\mathrm{text}}$.' Play around with the angle and justifications
to see how they work.
\section{Legends}
\label{sec-4}


The only legend option I find myself using is to put it on the bottom
of the plot:

\begin{verbatim}
ggplot( growth.270, aes(x=time,
                        y=OD,
                        col=strain,
                        group=plate_pos))+
    geom_line()+
    facet_grid(dextrose~ammonium, margins = TRUE)+
    theme_dramatic_black()+
    ggtitle("Gowth curves")+
    xlab("Time (seconds)")+
    theme(legend.position = "bottom")
\end{verbatim}


But note the changes made to the legend text and box above.

\end{document}