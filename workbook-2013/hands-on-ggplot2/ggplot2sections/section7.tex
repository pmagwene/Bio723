% Created 2013-08-21 Wed 17:11
\documentclass[11pt]{article}
\usepackage[utf8]{inputenc}
\usepackage[T1]{fontenc}
\usepackage{fixltx2e}
\usepackage{graphicx}
\usepackage{longtable}
\usepackage{float}
\usepackage{wrapfig}
\usepackage{soul}
\usepackage{textcomp}
\usepackage{marvosym}
\usepackage{wasysym}
\usepackage{latexsym}
\usepackage{amssymb}
\usepackage{hyperref}
\tolerance=1000
\providecommand{\alert}[1]{\textbf{#1}}

\title{stats}
\author{Colin Maxwell}
\date{21 August 2013}

\begin{document}

\maketitle

\setcounter{tocdepth}{3}
\tableofcontents
\vspace*{1cm}
One of the really powerful things about ggplot is that it will do
automatic calculations on your data and display the results. The way
that this is formalized is through a `statistic' that is calculated
before the data is passed to a geom. For geom$_{\mathrm{density}}$ and
geom$_{\mathrm{histogram}}$, this is done automatically in a logical way (ie the
frequency or the number is computed for each bin). However,
sometimes you want the ability to summarize your data yourself. In
this case you need to use `stat$_{\mathrm{summary}}$'. This is a function that
takes two basic arguments: a function and a string telling ggplot what
geom to use. The function is passed a vector of y values for every
unique x value. Because of this, stat$_{\mathrm{summary}}$ only works for discrete
x scales. Some examples will make this more clear.

\section{fun.y}
\label{sec-1}

Let's pretend that we want to compute some summary statistic on the
growth curve data. Maybe the data is noisy and we want to bin it
before visualizing it. The code below cuts the data into 20 bins and
plots the results. Note that there are now multiple y values for each
x value.

\begin{verbatim}
time.quantiles <- ecdf(growth.270$time)( growth.270$time )
growth.270$binned.time<- cut(time.quantiles, breaks = seq(0,1.1, 0.05))

ggplot(growth.270, aes(x=binned.time, y=OD, col = strain))+geom_point()
\end{verbatim}


A quick google search reminds me how to change the orientation of the
x axis text (see below for a brief discussion of themes).

\begin{verbatim}
ggplot(growth.270, aes(x=binned.time, y=OD, col = strain))+
    geom_point()+
    theme(axis.text.x = element_text(angle = 90, hjust = 1))
\end{verbatim}


The argument fun.y in stat$_{\mathrm{summary}}$ tells ggplot what function to call
for each X value. Since we defined a color scale, ggplot also first
splits the data by the strain. This is true in general: the statistic
is computed on data that has first been split by each of the aesthetic
mappings specified.

\begin{verbatim}
ggplot(growth.270, aes(x=binned.time, y=OD, col = strain))+
    theme(axis.text.x = element_text(angle = 90, hjust = 1))+
    stat_summary(fun.y = "mean", geom = "point")+
    scale_y_continuous("Mean OD")
\end{verbatim}


Of course, we could also choose a different function rather than the mean.

\begin{verbatim}
ggplot(growth.270, aes(x=binned.time, y=OD, col = strain))+
    theme(axis.text.x = element_text(angle = 90, hjust = 1))+
    stat_summary(fun.y = "median", geom = "point") +
    scale_y_continuous("Median OD")
\end{verbatim}


We can also define our own functions. The function must take a vector and return
a single number. In this case, I defined a function that will return 1
plus a small random number if the OD is over 0.5 and 0 plus a small
random number if it's less than 0.5. 

\begin{verbatim}
my.fxn <- function(x){
    if(x > 0.5){
        return( 1 + runif(n = 1, max = 0.1))
    }
    else{
        return( 0 + runif(n = 1, max=0.1))
    }
}

ggplot(growth.270, aes(x=binned.time, y=OD, col = strain))+
    theme(axis.text.x = element_text(angle = 90, hjust = 1))+
    stat_summary(fun.y = "my.fxn", geom = "point")+
      scale_y_continuous("Discretized OD")
\end{verbatim}


You might think that you could substitute the line geom for the point
geom, but you can't.

\begin{verbatim}
ggplot(growth.270, aes(x=binned.time, y=OD, col = strain))+
    theme(axis.text.x = element_text(angle = 90, hjust = 1))+
    stat_summary(fun.y = "median", geom = "line")
\end{verbatim}


The reason is that the geom is passed the split data that the statistic was computed on which
is length 1. This results in no lines being drawn since a line needs
two points. To correct this, we have to specify the group that we want
the lines to be drawn over:

\begin{verbatim}
ggplot(growth.270, aes(x=binned.time, y=OD, col = strain, group = strain))+
    theme(axis.text.x = element_text(angle = 90, hjust = 1))+
    stat_summary(fun.y = "median", geom = "line")
\end{verbatim}
\section{fun.data}
\label{sec-2}


There's a second way to call stat$_{\mathrm{summary}}$. Rather than using `fun.y',
you can specify `fun.data'. The difference between these two functions
is that fun.y will automatically create a variable called `y' based on
the summary statistic for every unique x value that is then passed to the geom specified. 
Look at the documentation and note that most geoms require a variable
called `y' to be defined in the data.frame.

On the other hand, fun.data gives you more flexibility in what data
variables are created, which means that you can use it with more
geoms. The downside to this is that you have to be
careful to specify what variables get created. So, the following code
won't work:

\begin{verbatim}
ggplot(growth.270, aes(x=binned.time, y=OD, col = strain, group = strain))+
    theme(axis.text.x = element_text(angle = 90, hjust = 1))+
    stat_summary(fun.data = "median", geom = "point")
\end{verbatim}


To see why this is, recall that in R vectors can have a `names'
attribute accessible by the `names' function. The code below
demonstrates that the result of the default `median' function is an
unnamed vector. 

\begin{verbatim}
test <- median( seq(0,100) )
print( names(test) )
\end{verbatim}


However, we can easily set the name to `y'.

\begin{verbatim}
names(test) <- "y"
print( names(test) )
\end{verbatim}


In fact, we can define a function that does so automatically, and this
function can then be passed to fun.data to give the behavior that we
wanted.

\begin{verbatim}
named.median <- function(x){
    out <- median(x)
    names(out) <- "y"
    out
}

print(names(named.median(1:100)))
\end{verbatim}


\begin{verbatim}
ggplot(growth.270, aes(x=binned.time, y=OD, col = strain))+
    theme(axis.text.x = element_text(angle = 90, hjust = 1))+
    stat_summary(fun.data = "named.median", geom = "point")
\end{verbatim}


In this case, it would be easier to just call
fun.y. However, some geoms require aesthetic variable mappings besides
just `y.' `geom$_{\mathrm{errorbar}}$', `geom$_{\mathrm{pointrange}}$', `geom$_{\mathrm{ribbon}}$', and `geom$_{\mathrm{crossbar}}$' all
require mappings called `ymax' and `ymin' in addition to `y'. In order
to use these geoms with our data, we need to define a function outputs
a named vector with all three of these names:

\begin{verbatim}

interquartile <- function(x){
    out <- quantile(x, probs = c(0.25, 0.5, 0.75))
    names(out) <- c("ymin", "y", "ymax")
    out
}

print( interquartile(1:100) )
\end{verbatim}


\begin{verbatim}
ggplot(growth.270, aes(x=binned.time, y=OD, col = strain))+
    theme(axis.text.x = element_text(angle = 90, hjust = 1))+
    stat_summary(fun.data = "interquartile", geom = "errorbar")
\end{verbatim}


You can then combine this plot with a line for a `trend with
errorbars' kind of graph (note the addition of the `group' aesthetic):

\begin{verbatim}
ggplot(growth.270, aes(x=binned.time, y=OD, col = strain, group = strain))+
    theme(axis.text.x = element_text(angle = 90, hjust = 1))+
    stat_summary(fun.data = "interquartile", geom = "errorbar")+
    stat_summary(fun.y = 'median', geom='line')
\end{verbatim}


You can also adjust any of the attributes of the geom that is
called through stat$_{\mathrm{summart}}$. For example, we can increase the line width and decrease the
errorbar width. (You can find out what you can do with each geom by
reading the documentation).

\begin{verbatim}
ggplot(growth.270, aes(x=binned.time, y=OD, col = strain, group = strain))+
    theme(axis.text.x = element_text(angle = 90, hjust = 1))+
    stat_summary(fun.data = "interquartile", geom = "errorbar", width=0.5)+
    stat_summary(fun.y = 'median', geom='line', lwd=1.5)
\end{verbatim}


Another cool geom is a ribbon:

\begin{verbatim}
ggplot(growth.270, aes(x=binned.time, y=OD, col = strain, group = strain))+
    theme(axis.text.x = element_text(angle = 90, hjust = 1))+
    stat_summary(fun.data = "interquartile", geom = "ribbon", alpha = 0.1)
\end{verbatim}


Of course, you can use other functions for fun.data. Two of the most
common ones are a 95\% confidence interval of the mean based on either
a normal distribution or a bootstrap. These are packaged with ggplot2
and are called `mean$_{\mathrm{cl}}$$_{\mathrm{normal}}$' and `mean$_{\mathrm{cl}}$$_{\mathrm{boot}}$' respectively. Note
that these are just run of the mill functions that you can call
yourself, they just happen to output named vectors:

\begin{verbatim}
print( mean_cl_boot( seq(0,100,1)) )
\end{verbatim}


\begin{verbatim}
print( mean_cl_normal( seq(0,100,1)) )
\end{verbatim}


In this case, I used the `pointrange' geom:

\begin{verbatim}
ggplot(growth.270, aes(x=binned.time, y=OD, col = strain, group = strain))+
    theme(axis.text.x = element_text(angle = 90, hjust = 1))+
    stat_summary(fun.data = "mean_cl_boot", geom = "pointrange")+
    stat_summary(fun.y = "mean", geom="line")
\end{verbatim}


You can also define a function to make a boxplot. It requires three
additional aesthetic mappings: `lower', `middle', and `upper'.

\begin{verbatim}

my.boxplot <- function(x){
    out <- c(mean(x), quantile(x, probs = c(0.05, 0.95, 0.25, 0.5, 0.75)))
    names(out) <- c("y", "ymin", "ymax", "lower", "middle", "upper")
    out
}

ggplot(growth.270, aes(x=binned.time, y=OD, col = strain))+
    theme(axis.text.x = element_text(angle = 90, hjust = 1))+
    stat_summary(fun.data = "my.boxplot", geom = "boxplot", fill = NA)
\end{verbatim}
\section{smoothed data}
\label{sec-3}


ggplot also gives you a nice way of plotting linear models with your
data. The function is called stat$_{\mathrm{smooth}}$. The default behavior is to
plot a loess regression:

\begin{verbatim}
ggplot( growth.270, aes(x=time,
                        y=OD,
                        col=strain,
                        group=plate_pos))+
    stat_smooth()
\end{verbatim}


However, you can actually specify any linear model you want. The
linear model type is specifed by the method argument. For example,
here's a normal linear model:

\begin{verbatim}
ggplot( growth.270, aes(x=time,
                        y=OD,
                        col=strain,
                        group=plate_pos))+
    stat_smooth( method = "lm", formula = y~x)+
    geom_point()
\end{verbatim}


You can also do logistic regression or anything else calling for a
generalized linear model. See the stat$_{\mathrm{smooth}}$ documentation for an example.

\end{document}