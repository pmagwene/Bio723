% Created 2013-08-21 Wed 17:10
\documentclass[11pt]{article}
\usepackage[utf8]{inputenc}
\usepackage[T1]{fontenc}
\usepackage{fixltx2e}
\usepackage{graphicx}
\usepackage{longtable}
\usepackage{float}
\usepackage{wrapfig}
\usepackage{soul}
\usepackage{textcomp}
\usepackage{marvosym}
\usepackage{wasysym}
\usepackage{latexsym}
\usepackage{amssymb}
\usepackage{hyperref}
\tolerance=1000
\providecommand{\alert}[1]{\textbf{#1}}

\title{geoms}
\author{Colin Maxwell}
\date{21 August 2013}

\begin{document}

\maketitle

\setcounter{tocdepth}{3}
\tableofcontents
\vspace*{1cm}

There are many different geoms you can use to plot your data. I'll
walk you through the most common below, but be sure to consult the
ggplot documentation to see the full range of tools in your toolbox.

\section{point}
\label{sec-1}


Probably the simplest geom to work with is the point. In ggplot land,
all geoms are encoded as R functions. The syntax used to add them to a
plot is simply a `+' sign:

\begin{verbatim}
ggplot(diamonds.sampled, aes(x=carat, y = price))+geom_point()
\end{verbatim}


geoms are modular, so you can add as many of them to a plot as you
want (as you'll see below). Each geom has particular aesthetics that
must be defined in the plot in order for it to be plotted. In the case
of geom$_{\mathrm{point}}$, the only required aesthetics are `x' and `y'. However,
pretty much any conceivable way you can think of to change a point can
be set as an aesthetic mapping (as we saw above).

Alternatively, if you just want all the points to be plotted in a
particular way, you can set any aspect of their aesthetics outside of
an aes function call:

\begin{verbatim}
ggplot(diamonds.sampled, aes(x=carat, y = price)) + geom_point(col = 'red', alpha = 0.1)
\end{verbatim}


The nice thing about ggplot is that once we have these pieces to play
with, we can start messing around to create a whole variety of plots.
For example, by setting the x value to discrete values and the y value to
continuous values, we get what people call a `strip plot':

\begin{verbatim}
ggplot( diamonds.sampled, aes(x=cut, y=price))+
    geom_point()
\end{verbatim}


However, this plot is pretty hard to read because there's so much
data. One solution to this is that points can be `jittered' to avoid overplotting:

\begin{verbatim}
ggplot( diamonds.sampled, aes(x=cut, y=price))+
    geom_point(position=position_jitter( width = 0.05, height = 0) )
\end{verbatim}


You have to play around to get the right amount of jitter:

\begin{verbatim}
ggplot( diamonds.sampled, aes(x=cut, y=price))+
    geom_point(position=position_jitter( width = 0.2, height = 0) )
\end{verbatim}


Setting the point transparency is also a good way to deal with overplotting

\begin{verbatim}
ggplot( diamonds.sampled, aes(x=cut, y=price))+
    geom_point(position=position_jitter( width = 0.1, height = 0),
               alpha = 0.1)
\end{verbatim}
\section{violin}
\label{sec-2}


Another approach to the problem of overplotting in strip charts is to
do a `violin' plot instead:

\begin{verbatim}
ggplot( diamonds.sampled, aes(x=cut, y=price))+
    geom_violin()
\end{verbatim}


As I mentioned above, ggplot is modular. So, there's nothing stopping
us from using more than one geom. For example, we could plot the
points in addition to the violin plot. Notice that I set the fill of
the violins and the color of the violins to make what I think is a
more pleasing plot:

\begin{verbatim}
ggplot( diamonds.sampled, aes(x=cut, y=price))+
    geom_violin(fill=NA, col='blue')+
    geom_point(position=position_jitter( width = 0.1, height = 0),
                 alpha = 0.1)
\end{verbatim}
\section{box}
\label{sec-3}


If your data are more or less normally distributed, a boxplot can be
substituted for a stripplot or a violin plot:

\begin{verbatim}
ggplot( diamonds.sampled, aes(x=cut, y=price))+
    geom_boxplot()
\end{verbatim}
\section{histograms}
\label{sec-4}

If your data aren't normally distributed, and if you care about the
absolute frequency of the data, a histogram is always nice.

\begin{verbatim}
ggplot( diamonds.sampled, aes(fill=cut, x=price))+
    geom_histogram()
\end{verbatim}


Make sure you play around with the number of bins you use:

\begin{verbatim}
ggplot( diamonds.sampled, aes(fill=cut, x=price))+
    geom_histogram(binwidth = 1000)
\end{verbatim}


There's two other ways to make a histogram: bins can be plotted beside
one another:

\begin{verbatim}
ggplot( diamonds.sampled, aes(fill=cut, x=price))+
    geom_histogram(position = "dodge")
\end{verbatim}


Or stacked on top for a kind of moving pie chart

\begin{verbatim}
ggplot( diamonds.sampled, aes(fill=cut, x=price))+
    geom_histogram(position = "fill")
\end{verbatim}
\section{density}
\label{sec-5}


If you don't want to determine the number of bins and you don't care
about the absolute number of observations for a particular value, a density plot is
sometimes appropriate:


\begin{verbatim}
ggplot( diamonds.sampled, aes(col=cut, x=price))+
    geom_density()
\end{verbatim}
\section{line}
\label{sec-6}


To demonstrate some aspects of ggplot, we'll use some data that I've
generated. The data is the optical density (OD) of some yeast cultures
seeded into a 96 well plate. There are three different yeast strains,
four different media conditions, and eight different initial dilutions
of the saturated culture into fresh media. First, we'll read in the
data and make some subsets for ease of plotting:

\begin{verbatim}
growth=read.csv("http://people.duke.edu/~csm29/growth_curves.csv", as.is =TRUE)
growth$ammonium <- paste(sapply(strsplit(growth$media, ""), "[", 1), "ammonium")
growth$dextrose <- paste(sapply(strsplit(growth$media, ""), "[", 3), "dextrose")

growth.PMY1529 <- subset(growth, (strain == "PMY1529") & (initial_dilution == 270))
growth.270 <- subset(growth, (initial_dilution == 270))
\end{verbatim}

Since there are four media types, plotting the subset of the data that
contains a single dilution amount and a single strain has four obvious
lines corresponding to the media:

\begin{verbatim}
ggplot( growth.PMY1529, aes(x=time, y=OD))+geom_point()
\end{verbatim}


However, if we just try to substitute a line for a point, we get
something that looks unintelligible:

\begin{verbatim}
ggplot( growth.PMY1529, aes(x=time, y=OD))+geom_line()
\end{verbatim}


This is because you must specify what `group' of values that ggplot
should connect. To see this, we'll take an additional subset of the
data to restrict ourselves to just one media type. As long as there's
only one set of data to deal with, our approach worked OK:

\begin{verbatim}
ggplot(subset(growth.PMY1529, media == "HAHD"),
       aes(x=time, y=OD))+geom_line()
\end{verbatim}


The way you tell ggplot to connect different sets of points in the
data is to set the `group' aesthetic:

\begin{verbatim}
ggplot( growth.PMY1529, aes(x=time, y=OD, group=media))+geom_line()
\end{verbatim}


Instead of setting the group aesthetic, we could also just set the
color to be the media type. ggplot is smart enough to connect like
colored points when using geom$_{\mathrm{line}}$.

\begin{verbatim}
ggplot( growth.PMY1529, aes(x=time, y=OD, col=media))+geom_line()
\end{verbatim}


However, if we wanted to split by all media types, but color by only the
dextrose concentration, we could specify both a color and a group:

\begin{verbatim}
ggplot( growth.PMY1529, aes(x=time, y=OD, col=dextrose, group=media))+geom_line()
\end{verbatim}


And, of course, you can specify mappings until the plot gets hard to read.

\begin{verbatim}
ggplot( growth.270, aes(x=time,
                        y=OD,
                        col=strain,
                        lty=dextrose,
                        group=plate_pos))+geom_line()
\end{verbatim}


Strangely, you can't do a similar trick with the boxplots. For
example, say that I wanted each `x' to be colored by `color' and
grouped by `clarity:'

\begin{verbatim}
ggplot( diamonds.sampled, aes(x=cut, y=price, col=color, group=clarity))+
    geom_boxplot()
\end{verbatim}


The reasons why this is the case are complicated, but it's good to be
aware of the fact. The best you can do have both `color' and `x':

\begin{verbatim}
ggplot( diamonds.sampled, aes(x=cut, y=price, col=color))+
    geom_boxplot()
\end{verbatim}
\section{tile}
\label{sec-7}


Sometimes you have two variables that are evaluated for a third
variable on an evenly spaced grid (think of a fitness landscape). In
this case, you can use the `tile' geom in order to color tiles
according to the value at that coordinate:

\begin{verbatim}
#This is a function to make a dataframe of a two variabled function
pp <- function (n,r=4) {
 x <- seq(-r*pi, r*pi, len=n)
 df <- expand.grid(x=x, y=x)
 df$r <- sqrt(df$x^2 + df$y^2)
 df$z <- cos(df$r^2)*exp(-df$r/6)
 df
}
#Note that the aesthetic set is the 'fill' not the 'color'
ggplot(pp(100), aes(x=x,y=y, fill=z))+geom_tile()
\end{verbatim}

\end{document}