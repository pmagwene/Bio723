% Created 2013-08-21 Wed 17:10
\documentclass[11pt]{article}
\usepackage[utf8]{inputenc}
\usepackage[T1]{fontenc}
\usepackage{fixltx2e}
\usepackage{graphicx}
\usepackage{longtable}
\usepackage{float}
\usepackage{wrapfig}
\usepackage{soul}
\usepackage{textcomp}
\usepackage{marvosym}
\usepackage{wasysym}
\usepackage{latexsym}
\usepackage{amssymb}
\usepackage{hyperref}
\tolerance=1000
\providecommand{\alert}[1]{\textbf{#1}}

\title{ggplot}
\author{Colin Maxwell}
\date{21 August 2013}

\begin{document}

\maketitle

\setcounter{tocdepth}{3}
\tableofcontents
\vspace*{1cm}
Probably the biggest obstacle to learning ggplot-style graphics is
  that it requires learning a different syntax for plotting. The
  following sections break down a ggplot into it's component parts.

We've already seen the function `ggplot' itself. The first argument is
always a data frame. The data frame is the one that ggplot will use to
look for all the mappings that you define in the subsequent pieces of
the plot. The nice thing about this is that there is no need to use
the dollar sign notation. (You can get a similar behavior in base
plots by specifying the `data' argument)

The second argument is always a function called `aes.' aes takes named
arguments. Each argument name is the `aesthetic' that you want mapped
to a particular column in the data. The way you specify the mapping is
by writing the aesthetic (ie size) `=' the column name. (Note that the column names aren't in quotes) For example, if we want the diamond's caret to be on the x axis and the diamond's price to be on the y axis, we would write:

\begin{verbatim}
ggplot(diamonds.sampled, aes(x=carat, y = price))
\end{verbatim}

If you call just this function, nothing will be plotted because we haven't told ggplots how to
display the aesthetic mappings we've made. To do this, we need a geom.

\end{document}