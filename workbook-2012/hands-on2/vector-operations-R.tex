%!TEX root = ./workbook-2011.tex

\section{Vector Operations in R}

As you saw last week R vectors support basic arithmetic operations that
correspond to the same operations on geometric vectors. For example:
%
\begin{R}
> x <- 1:15
> y <- 10:24
> x
 [1]  1  2  3  4  5  6  7  8  9 10 11 12 13 14 15
> y
 [1] 10 11 12 13 14 15 16 17 18 19 20 21 22 23 24
> x + y             # vector addition
 [1] 11 13 15 17 19 21 23 25 27 29 31 33 35 37 39
> x - y             # vector subtraction
 [1] -9 -9 -9 -9 -9 -9 -9 -9 -9 -9 -9 -9 -9 -9 -9
> x * 3             # multiplication by a scalar
 [1]  3  6  9 12 15 18 21 24 27 30 33 36 39 42 45 
\end{R}
%
R also has an operator for the dot product, denoted \lstinline!%*%!.
This operator also designates matrix multiplication, which we will
discuss next week. By default this operator returns an object of the R
matrix class. If you want a scalar (or the R equivalent of a scalar,
i.e.~a vector of length 1) you need to use the \lstinline!drop()!
function.

\begin{R}
> z <- x %*% x
> class(z)      # note use of class() function
[1] "matrix"
> z
     [,1]
[1,] 1240
> drop(z)
[1] 1240
\end{R}

\begin{assignment}
In R, use the dot product operator and the
\lstinline!acos()! function to calculate the angle (in radians) between
the vectors \lstinline!x = [-3, -3, -1, -1, 0, 0, 1, 2, 2, 3]! and
\lstinline!y = [-8, -5, -3, 0, -1, 0, 5, 1, 6, 5]!.
\end{assignment}

