

\section{Vector Geometry of Correlation and Regression}

Let's return to our use of the dot product to explore the relationship between variables. First let's add a function to our module, |vecgeom.R|, to calculate the cosine of the angle between to vectors.
\begin{R}
# add to vecgeom.R

vec.cos <- function(x,y) {
  # Calculate the cos of the angle between vectors x and y
  len.x <- veclength(x)
  len.y <- veclength(y)
  return( (x %*% y)/(len.x * len.y) )
}
\end{R}


We can then use this function to examine the relationships between the variables in the iris dataset. For now let's just work with the \Species{I.~setosa} specimens. Read the help file for |subset()|.
\begin{R}
> setosa <- subset(iris, Species == 'setosa', select = -Species)
> dim(setosa)
[1] 50  4
> names(setosa)
[1] "Sepal.Length" "Sepal.Width"  "Petal.Length" "Petal.Width"
\end{R}
%
Often times it's useful to look at many bivariate relationships simultaneously. The |pairs()| function allows you to do this:
%
\begin{R}
> pairs(setosa)
\end{R}
%
\begin{figure}[htbp]
\centering
\includegraphics[width=0.5\columnwidth]{./figures/hands-on2/pairs-output.pdf}
\caption{Output of the \lstinline!pairs()! function for the \Species{I.~setosa} specimens in the \lstinline|iris| dataset.}
\end{figure}


First we'll center the setosa dataset using the |scale()| function. |scale()| has two logical arguments |center| and |scale|. By default both are |TRUE| which will center \emph{and} scale the variables. But for now we just want to center the data. |scale()| returns a matrix object so we use the |data.frame| function to cast the object back to a data frame.
\begin{R}
> source("/Users/pmagwene/Downloads/vecgeom.R")
> ctrd <- scale(setosa,center=T,scale=F)
> class(ctrd)
[1] "matrix"
> names(ctrd)
NULL
> ctrd <- data.frame(scale(setosa,center=T,scale=F))
> class(ctrd)
[1] "data.frame"
> names(ctrd)
[1] "Sepal.Length" "Sepal.Width"  "Petal.Length" "Petal.Width"
> vec.cos(ctrd$Sepal.Length, ctrd$Sepal.Width)
          [,1]
[1,] 0.7425467
> vec.cos(ctrd$Sepal.Length, ctrd$Petal.Length)
          [,1]
[1,] 0.2671758
> vec.cos(ctrd$Sepal.Length, ctrd$Petal.Width)
          [,1]
[1,] 0.2780984
\end{R}

Consider the values above in the context of the scatter plots you generated with the |pairs()| function; and then recall that for mean-centered variables, $\mathsf{cor}(X,Y) = r_{XY} = \cos \theta = \frac{\vec{x} \cdot \vec{y}}{\vert \vec{x}\vert \vert \vec{y} \vert}$.  So our |vec.cos()| function, when applied to centered data, is equivalent to calculating the correlation between $x$ and $y$.  Let's confirm this using the built in |cor()| function in R:
\begin{R}
> cor(setosa$Sepal.Length, setosa$Sepal.Width)
[1] 0.7425467
> cor(setosa)  # called like this will calculate all pairwise correlations
             Sepal.Length Sepal.Width Petal.Length Petal.Width
Sepal.Length    1.0000000   0.7425467    0.2671758   0.2780984
Sepal.Width     0.7425467   1.0000000    0.1777000   0.2327520
Petal.Length    0.2671758   0.1777000    1.0000000   0.3316300
Petal.Width     0.2780984   0.2327520    0.3316300   1.0000000
\end{R}


\subsection{Bivariate Regression in R}

R has a flexible built in function, |lm()| for fitting linear models. Bivariate regression is the simplest case of a linear model.
%
\begin{R}
> setosa.lm <- lm(Sepal.Width ~ Sepal.Length, data=setosa)
> class(setosa.lm)
[1] "lm"
> names(setosa.lm)
 [1] "coefficients"  "residuals"     "effects"       "rank"
 [5] "fitted.values" "assign"        "qr"            "df.residual"
 [9] "xlevels"       "call"          "terms"         "model"
> coef(setosa.lm)
 (Intercept) Sepal.Length 
  -0.5694327    0.7985283 
\end{R}
The function |coef()| will return the intercept and slope of the line representing the bivarariate regression. For a more complete summary of the linear model you've fit use the |summary()| function:
\begin{R}
> summary(setosa.lm)

Call:
lm(formula = Sepal.Width ~ Sepal.Length, data = setosa)

Residuals:
     Min       1Q   Median       3Q      Max
-0.72394 -0.18273 -0.00306  0.15738  0.51709

Coefficients:
             Estimate Std. Error t value Pr(>|t|)
(Intercept)   -0.5694     0.5217  -1.091    0.281
Sepal.Length   0.7985     0.1040   7.681 6.71e-10 ***
---
Signif. codes:  0 ‘***’ 0.001 ‘**’ 0.01 ‘*’ 0.05 ‘.’ 0.1 ‘ ’ 1

Residual standard error: 0.2565 on 48 degrees of freedom
Multiple R-squared: 0.5514, Adjusted R-squared: 0.542
F-statistic: 58.99 on 1 and 48 DF,  p-value: 6.71e-10
\end{R}
%
As demonstrated above, the |summary()| function spits out key diagnostic information about the model we fit.  Now let's create a plot illustrating the fit of the model.
%
\begin{R}
> plot(Sepal.Width ~ Sepal.Length, data=setosa, xlab="Sepal Length (cm)", ylab="Sepal Width (cm)", main="Iris setosa")
> abline(setosa.lm, col='red', lwd=2, lty=2)  # see ?par for info about lwd and lty
\end{R}
Your output should resemble the figure below. Note the use of the function \lstinline!abline()! to plot the regression
line. Calling \lstinline!plot()! with an object of class \lstinline!lm!
shows a series of diagnostic plots. Try this yourself.
%
\begin{figure}[htbp]
\centering
\includegraphics[width=0.5\columnwidth]{./figures/hands-on2/regression.pdf}
\caption{Linear regression of Sepal Width on Sepal Length for \Species{I.~setosa}.}
\end{figure}

% \begin{assignment}
% Write your own regression function (i.e.~your
% code shouldn't refer to the built in regression functions) for mean
% centered vectors in R. The function will take as it's input two vectors,
% $\vec{x}$ and $\vec{y}$. The function should return:

% \begin{enumerate}[1.]
% \item
%   a list containing the mean-centered versions of these vectors
% \item
%   the regression coefficient $b$ in the mean centered regression
%   equation $\vec{\widehat{y}} = b\vec{x}$
% \item
%   the coefficient of determination, $R^2$
% \end{enumerate}
% Demonstrate your regression function by using it to carry out
% regressions of Sepal.Length on Sepal.Width separately for the `versicolor'
% and `virginica' specimens from the iris data set. Include |ggplot| created plots in which you use the
% \lstinline!geom_point()! and \lstinline!geom_abline()! functions to illustrate your
% calculated regression line. To test your function, compare your regression coefficients and coefficient of determination to the same values returned by the built in |lm()| function.

% \end{assignment}



